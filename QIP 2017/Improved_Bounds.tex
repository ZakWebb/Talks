%%%%%%%%%%%%%%%%%%%%%%%%%%%%
% LaTeX poster template
% Created by Nathaniel Johnston
% August 2009
% http://www.nathanieljohnston.com/2009/08/latex-poster-template/
%%%%%%%%%%%%%%%%%%%%%%%%%%%%%%%%%%%%%%

\PassOptionsToPackage{gray}{xcolor}
\documentclass{beamer}
\usepackage[orientation=landscape,size=a0]{beamerposter}
\usepackage{graphicx}			% allows us to import images

%-----------------------------------------------------------
% The packages used in our latex file
%-----------------------------------------------------------

\usepackage[T1]{fontenc}
\usepackage{svg}
\usepackage{amsmath,amssymb,amsthm}
\usepackage{mathtools}
\mathtoolsset{centercolon}
\usepackage{tikz}
\makeatletter
\protected\def\tikz@nonactivecolon{\ifmmode\mathrel{\mathop\ordinarycolon}\else:\fi} 
\makeatother
\usetikzlibrary{decorations.pathreplacing}
%\usepackage{pgfplots}


%\usepackage{subfigure}
\usetikzlibrary{decorations.markings}
\usepackage{setspace}

\setlength{\parskip}{\baselineskip} 
%\usepackage{verbatim}
%\usepackage[normalem]{ulem} 
%\usepackage{thmtools}
%\usepackage{color}
%\usepackage[english]{babel}


%-----------------------------------------------------------
% Define the column width and poster size
% To set effective sepwid, onecolwid and twocolwid values, first choose how many columns you want and how much separation you want between columns
% The separation I chose is 0.024 and I want 4 columns
% Then set onecolwid to be (1-(4+1)*0.024)/4 = 0.22
% Set twocolwid to be 2*onecolwid + sepwid = 0.464
%-----------------------------------------------------------

\newlength{\sepwid}
\newlength{\onecolwid}
\newlength{\twocolwid}
\newlength{\threecolwid}
\setlength{\sepwid}{0.04\paperwidth}
\setlength{\onecolwid}{0.28\paperwidth}
\setlength{\twocolwid}{0.6\paperwidth}
\setlength{\threecolwid}{.92\paperwidth}
\setlength{\topmargin}{-0.5in}
\usetheme{confposter}
\usepackage{exscale}

%-----------------------------------------------------------
% The next part fixes a problem with figure numbering. Thanks Nishan!
% When including a figure in your poster, be sure that the commands are typed in the following order:
% \begin{figure}
% \includegraphics[...]{...}
% \caption{...}
% \end{figure}
% That is, put the \caption after the \includegraphics
%-----------------------------------------------------------

\usecaptiontemplate{
\small
\structure{\insertcaptionname~\insertcaptionnumber:}
\insertcaption}

%-----------------------------------------------------------
% Define colours (see beamerthemeconfposter.sty to change these colour definitions)
%-----------------------------------------------------------

\newcommand{\Norm}[1]{\left\|{#1}\right\|}
\newcommand{\norm}[1]{\|{#1}\|}




\setbeamercolor{block title}{fg=ngreen,bg=white}
\setbeamercolor{block body}{fg=black,bg=white}
\setbeamercolor{block alerted title}{fg=white, bg=dblue!70}
\setbeamercolor{block alerted body}{fg=black, bg=dblue!50}

%%%%%%%%%%%%%%%%%%%%%%%%%%%%%%%%%%%%%%
% Miscellaneous Macros

\newcommand{\sbra}[1]{\langle{#1}|}
\newcommand{\sket}[1]{|{#1}\rangle}
\newcommand{\sbraket}[2]{\langle{#1}|{#2}\rangle}
\newcommand{\ceil}[1]{\lceil{#1}\rceil}
\newcommand{\CC}{\mathbb{C}}
\newcommand{\II}{\mathbb{I}}
\newcommand{\KK}{\mathbb{K}}
\newcommand{\QQ}{\mathbb{Q}}
\newcommand{\RR}{\mathbb{R}}
\newcommand{\ZZ}{\mathbb{Z}}
\newcommand{\posint}{\ZZ^+}
\renewcommand{\sc}{\mathrm{sc}}
\newcommand{\bc}{\mathrm{bc}}
\newcommand{\ph}{\mathrm{ph}}
\newcommand{\OO}{\mathcal{O}}

\DeclareMathOperator{\spn}{span}


%-----------------------------------------------------------
% Name and authors of poster/paper/research
%-----------------------------------------------------------

\title{Universal computation by multiparticle quantum walk\\ with improved error bounds}
\author{Zak Webb} 

\institute{
  Institute for Quantum Computing, University of Waterloo\\
  Department of Physics \& Astronomy, University of Waterloo\\
  Department of Computer Science, University of Texas at Austin}
  
  
  \begin{document}

%----------------------------------------------------------
% Including the Logos
%----------------------------------------------------------

\addtobeamertemplate{headline}{} 
{
\begin{tikzpicture}[remember picture,overlay] 
  \node [shift={(12 cm,-5cm)}] at (current page.north west) {\includegraphics[height=10cm]{UW_Logo_BW}};
\node [shift={(8.3cm,-10cm)}] at (current page.north west) {\includegraphics[height=5.66cm]{IQC_Logo}};
\node [shift={(-14.5 cm,-4.75cm)}] at (current page.north east) {\includegraphics[height=8cm]{UT_Logo_BW}}; 

\end{tikzpicture} 
}


\tikzset{global scale/.style={
    scale=#1,
    every node/.style={scale=#1}
  }
}
%-----------------------------------------------------------
% Start the poster itself
%-----------------------------------------------------------

\begin{frame}[t]
\begin{columns}[t]
\begin{column}{\sepwid}\end{column}

\begin{column}{\onecolwid}
  \begin{alertblock}{Abstract}
    Quantum walk has proven itself a powerful resource for quantum computing.  In the single particle model, Childs \cite{Chi09} showed that on an exponential sized graph, quantum walk is universal for quantum computation.  Childs, Gosset, and Webb \cite{MPQW} then showed the universality of multiparticle quantum walk on polynomially sized graphs.  In this work, we improve the work of \cite{MPQW} by improving their error bounds.

    In particular, \cite{MPQW} used graph scattering with square wavepackets to encode the qubits.  By encoding each qubit via a dual-rail encoding, they were able to use single particle and particular two-particle scattering behavior to analyze the many-particle behavior with high precision.  However, to ease analysis they used square wavepackets to encode each qubit.  In this work, we use the same construction as in \cite{MPQW}, but use cutoff Gaussian wavepackets instead.  This then leads to a near quadratic improvement in the resulting error bounds.
  \end{alertblock}
  
  \begin{block}{Single particle quantum walk}
    For our purposes, we will use continuous time quantum walk.  For a given graph $G$, this means that the time evolution will simply be determined by taking the adjacency matrix of $G$ as the Hamiltonian for the system:
    \[
      U(t) = \exp(- i A(G) t).
    \]
    With this, the underlying Hilbert space is spanned by vertex states $\{\sket{v}, v\in V(G)\}$.

    With this time evolution, we will be interested in understanding the eigenstates of $A(G)$.  As a particular example, if $G$ is an infinite path (think of a discretization of the real line), we have that the eigenstates are of the so-called ``momentum'' states:
    \[
      \sket{\tilde{k}} = \sum_{x\in \ZZ} e^{-i k x} \sket{x} \qquad k\in [-\pi,\pi).
    \]
    These states move at a speed that depends on $k$, in a similar manner to a one-dimensional free particle.
    
    

    
  \end{block}


  \begin{block}{Multiparticle quantum walk}
    To generalize quantum walk to multiple particles, we will assume that there is some finite-range interaction between particles, that only depends on the number of particles on two vertices, and the distance between the two vertices.  Namely, we will assume that the $N$-particle Hilbert space for a quantum walk on a graph $G$ is spanned by states of the form
    \[
      \big\{\sket{v_1,v_2,\cdots,v_N}: v_i\in V(G)\big\}.
    \]
    The Hamiltonian governing the time evolution is then given by
    \[
      H_G^N = H_{\text{move}} + H_{\text{int}}
    \]
    where
    \[
      H_{\text{move}} = A(G)\otimes \II\otimes \II \otimes \cdots \otimes \II + \cdots + \II \otimes \cdots\otimes \II \otimes A(G)
    \]
    and
    \[
      H_{\text{int}} = \sum_{d=0}^{R} \sum_{\substack{u,v\in V(G)\\d(u,v)=d}} U_{d}(\hat{n}_u,\hat{n}_v)
    \]
    where $R$ is the maximum range of interaction, $\hat{n}_u$ counts the number of particles located on vertex $u$, and $U_d$ are symmetric polynomials.

    Note that this framework easily includes such things as the Bose-Hubbard model and nearest neighbor interactions.  Similarly, as the Hamiltonian is symmetric, we can restrict ourselves to bosonic or fermionic particles without problems.
    
    
  \end{block}

\end{column}

%%%%% 
\begin{column}{\sepwid}\end{column}

\begin{column}{\onecolwid}
  \begin{block}{Graph scattering}
    We can use our knowledge of the eigenstates on long paths to understand the dynamics when we attach long paths to finite sized graphs; most eigenstates take the form of scattering states.
    
    The particular scattering behavior depends on the finite graph to which the long paths are attached, the number of paths, and the momentum of the incoming particle, but the scattering behavior is relatively simple to determine.
    
    Moreover, while the eigenstates of the system do not evolve in time, we prove that the time evolution of a truncated Gaussian wavepacket moves forward at the correct speed, hits the finite graph, and scatters off of the graph with the amplitudes as determined by the scattering eigenstate to high precision.  As such, our intuition for these scattering states are well-founded.


\begin{figure}
\centering
\begin{tikzpicture}[
  scale=2,
  label distance=-5.5pt,
  thin,
  vertex/.style={circle,draw=black,fill=black,inner sep=1.25pt,
    minimum size =0mm},
  attach/.style={circle,draw=black,fill=white,inner sep=1.25pt,
    minimum size =0mm},
  dots/.style={circle,fill=black,inner sep=.5pt,
    minimum size= 0pt},
  every text node part/.style={font=\footnotesize}]

  \node at (.15, .63) [rectangle,fill=white] {$(1,1)$};
  \node at (.48, .17) [rectangle,fill=white,inner sep=0pt] {$(1,2)$};
  \node at (.22,-.5) [rectangle,fill=white] {$(1,N)$};

  \draw[thick] (15:1) arc (15:335:1);
  \draw[densely dotted] (15:1)  arc (15:5:1);
  \draw[densely dotted] (-15:1)  arc (-15:-25:1);

  \node at (-.42,.12) [rectangle] {\normalsize ${\hat{G}}$};

  \foreach \x  in {50, 25,-35}{
    \draw (\x:1cm) -- (\x:3.15cm);
    \draw (\x:2.9cm) -- (\x:3.4cm) [densely dotted];
  }
  
  \node at (50:1)[attach]{};
  \node at (50:1.66)[vertex,label=150:{$(2,1)$}]{};
  \node at (50:2.33)[vertex,label=150:{$(3,1)$}]{};
  \node at (50:3)[vertex,label=150:{$(4,1)$}]{};

  \node at (25:1)[attach]{};
  \node at (25:1.66)[vertex]{};
  \node at (25:2.33)[vertex]{};
  \node at (25:3)[vertex]{};
  
  \node at (12:1.65) {$(2,2)$};
  \node at (15:2.5) {$(3,2)$};
  \node at (17.5:3.35) {$(4,2)$};

  \node at (-35:1)[attach]{};
  \node at (-35:1.66)[vertex,label=60:{$(2,N)$}]{};
  \node at (-35:2.33)[vertex,label=60:{$(3,N)$}]{};
  \node at (-35:3)[vertex,label=60:{$(4,N)$}]{};

  \node at (-2.5:2.9)[dots] {};
  \node at (2.5:2.9)[dots] {};
  \node at (-7.5:2.9)[dots] {};
\end{tikzpicture}
\end{figure}    




  \end{block}

  \begin{block}{Two-particle scattering}

	General two-particle scattering events are too difficult to analyze analytically, but on restricted graphs the understanding is possible.  In particular, symmetric considerations allow us to determine the eigenstates of two-particles on a long path.  Assuming that the interaction is constant range, most eigenestates take a scattering form with the two particles having momentum states moving toward each other, interacting, and then moving apart with some applied phase after interacting.  Both momenta are unaffected, and the only result of the interaction is to apply a phase to the state.   

  \end{block}

  \begin{block}{Universal Construction}
    Using the idea of momentum states, we can encode each qubit via the location of a particle in a dual-rail encoding.  In particular, each qubit will have an associated particle moving at a given momentum, and a long path for each computational basis state.  As such, an $n$-qubit computation will be encoded with $n$ particles and $2n$ long paths.  Note that in this work we use truncated Gaussian wavepackets.
    
    Each single qubit gate will be applied via a single particle scattering events.  These gadgets will depend on the momentum, but there exist universal sets for several momenta.
    
    Each entangling gate will be a controlled-$Z$ rotation about some angle that depends on particle momentum.  They will be applied by routing two particles toward each other if both particles are on the $1$-paths.  In this manner, they will acquire the interacting phase if and only if both particles are in the computational $1$ state.
    
	A quantum circuit is simulated by applying each scattering event in series, with long paths between each event to ensure that the wavepackets are all located a distance from the scattering gadgets.
    
    Measurement in the computational basis is then done by determining the location of all particles.

\begin{figure}
\centering
\begin{tikzpicture}
  [ scale = .8,
    xscale = 0.8,
    inner/.style={circle,draw=black!100,fill=black!100,inner sep=1pt},
    attach/.style={circle,draw=black!100,fill=black!0,thin,inner sep=.7pt},
    switch/.style={circle,draw=black!100,fill=black!50,thin,inner sep=1.25 pt},
    cross/.style={draw=white,double=black,ultra thick},
    splitter/.style={circle,fill=white,draw=black!100,inner sep=.8mm},
    had/.style={rectangle,draw=black!100,fill=white,rounded corners,thick,minimum size = 1},
    verts/.style={circle,fill=black,inner sep =.6pt,minimum size = 0pt},
    vert/.style={circle,fill=black,inner sep=0.5pt,minimum size=0pt},
    decoration={markings, 
      mark= between positions 0 and 1 step 1.2 mm with{
        \node[vert] at (0,0){};
      }}
    ]
  \node (mao) at (20,0) [attach,label=right:$0_{3,\text{out}}$] {};

  \node (mbo) at (20,-1) [attach,label=right:$1_{3,\text{out}}$] {};
 
  \node (l2bo) at (20,1.5) [attach,label=right:$1_{2,\text{out}}$] {};

  \node (l2ao) at (20,2.5)   [attach,label=right:$0_{2,\text{out}}$] {};

  \node (l1bo) at (20,4) [attach,label=right:$1_{1,\text{out}}$] {};

  \node (l1ao) at (20,5) [attach,label=right:$0_{1,\text{out}}$] {};
 
  \node (l1aa) at (-20,5) [attach,label=left:$0_{1,\text{in}}$] {};
  \node (l1ba) at (-20,4) [attach,label=left:$1_{1,\text{in}}$] {};
  \node (l2aa) at (-20,2.5) [attach,label=left:$0_{2,\text{in}}$] {};
  \node (l2ba) at (-20,1.5) [attach,label=left:$1_{2,\text{in}}$] {};
  \node (maa)  at (-20,0) [attach,label=left:$0_{3,\text{in}}$] {};
  \node (mba)  at (-20,-1) [attach,label=left:$1_{3,\text{in}}$] {};

  \node (bl1) at (-11,-1) [splitter] {};
  \node (bl2) at (0, -1) [splitter] {};
  \node (bl3) at (9, -1) [splitter] {};

  \node (br1) at (-9,-1) [splitter] {};
  \node (br3) at (11, -1) [splitter] {};

  \node (tl1) at (-11, 4) [splitter] {};
  \node (tl2) at (0 ,1.5) [splitter] {};
  \node (tl3) at (9,  4) [splitter] {};

  \node (tr1) at (-9, 4) [splitter] {};
  \node (tr3) at (11,  4) [splitter] {};

  \begin{scope}[thin]
  \draw[postaction={decorate}] (l1aa) -- (l1ao);
  \draw[postaction={decorate}] (l2aa) -- (14.25,2.5);
  \draw[postaction={decorate}] (15.75,2.5) -- (l2ao);
  \draw[postaction={decorate}] (l1ba) -- (tl1.west);
  \draw[postaction={decorate}] (tl1.east) -- (tr1.west);
  \draw[postaction={decorate}] (tr1.east) -- (tl3.west);
  \draw[postaction={decorate}] (tl3.east) -- (tr3.west);
  \draw[postaction={decorate}] (tr3.east) -- (l1bo);
  \draw[postaction={decorate}] (l2ba) -- (tl2.west);
  \draw[white,line width=5pt] (0.5,1.5) to[out=0,in=180] (2.5,-1);
  \draw[postaction={decorate}] (tl2.east) -- (0.5,1.5) 
      to[out=0,in=180] (2.5,-1) -- (4.25,-1);
  \draw[postaction={decorate}] (5.75,-1) -- (bl3.west);
  \draw[postaction={decorate}] (15.75,1.5) -- (l2bo);
  \draw[postaction={decorate}] (maa)  -- (-15.75,0);
  \draw[postaction={decorate}] (-14.25,0) -- (-5.75,0);
  \draw[postaction={decorate}] (-4.25,0) -- (4.25,0);
  \draw[postaction={decorate}] (5.75,0) -- (14.25,0);
  \draw[postaction={decorate}] (15.75,0) -- (mao);
  \draw[postaction={decorate}] (mba)  -- (-15.75,-1);
  \draw[postaction={decorate}] (-14.25,-1) -- (bl1.west);
  \draw[postaction={decorate}] (bl1.east) -- (br1.west);
  \draw[postaction={decorate}] (br1.east) -- (-5.75,-1);
  \draw[postaction={decorate}] (-4.25,-1) -- (bl2.west);
  \draw[white,line width=5pt] (0.5,-1) to[out=0,in=180] (2.5,1.5);
  \draw[postaction={decorate}] (bl2.east) -- (0.5,-1) 
      to[out=0,in=180] (2.5,1.5) -- (14.25,1.5);
  \draw[postaction={decorate}] (bl3.east) -- (br3.west);
  \draw[postaction={decorate}] (br3.east) -- (14.25,-1);
  \draw[postaction={decorate}] (15.75,-1) -- (mbo);
  \end{scope}

  \begin{scope}[draw=white, line width = 5 pt]
  \draw (bl1.north) -- (tl1.south);
  \draw (bl1.north) -- (tl1.south);
  \draw (bl2.north) -- (tl2.south);
  \draw (bl3.north) -- (tl3.south);
  
  \draw (br1.north) -- (tr1.south);
  \draw (br3.north) -- (tr3.south);
  \end{scope}

  \begin{scope}[thin]
  \draw[postaction={decorate}] (bl1.north) -- (tl1.south);
  \draw[postaction={decorate}] (bl2.north) -- (tl2.south);
  \draw[postaction={decorate}] (bl3.north) -- (tl3.south);
  
  \draw[postaction={decorate}] (br1.north) -- (tr1.south);
  \draw[postaction={decorate}] (br3.north) -- (tr3.south);
  \end{scope}
 
  \draw (br1.west) to[out=0,in=-90] (br1.north);
  \draw (br3.west) to[out=0,in=-90] (br3.north);

  \draw (tr1.south) to[out=90,in=180] (tr1.east);
  \draw (tr3.south) to[out=90,in=180] (tr3.east);

  \draw (tl1.west) to[out=0,in=90] (tl1.south);
  \draw (tl2.west) to[out=0,in=90] (tl2.south);
  \draw (tl3.west) to[out=0,in=90] (tl3.south);

  \draw (bl1.north) to[out=-90,in=180] (bl1.east);
  \draw (bl2.north) to[out=-90,in=180] (bl2.east);
  \draw (bl3.north) to[out=-90,in=180] (bl3.east);


  \draw[line width=.9pt] (bl1.west) to[out=0,in=-90] (bl1.north);
  \draw[line width=.9pt] (bl2.west) to[out=0,in=-90] (bl2.north);
  \draw[line width=.9pt] (bl3.west) to[out=0,in=-90] (bl3.north);

  \draw[line width=.9pt] (br1.east) to[out=180,in=-90] (br1.north);
  \draw[line width=.9pt] (br3.east) to[out=180,in=-90] (br3.north);

  \draw[line width=.9pt] (tr1.south) to[out=90,in=0] (tr1.west);
  \draw[line width=.9pt] (tr3.south) to[out=90,in=0] (tr3.west);

  \draw[line width=.9pt] (tl1.east) to[out=180,in=90] (tl1.south);
  \draw[line width=.9pt] (tl2.east) to[out=180,in=90] (tl2.south);
  \draw[line width=.9pt] (tl3.east) to[out=180,in=90] (tl3.south);

  \draw[line width=.3pt,draw=white] (bl1.west) to[out=0,in=-90] (bl1.north);
  \draw[line width=.3pt,draw=white] (bl2.west) to[out=0,in=-90] (bl2.north);
  \draw[line width=.3pt,draw=white] (bl3.west) to[out=0,in=-90] (bl3.north);

  \draw[line width=.3pt,draw=white] (br1.east) to[out=180,in=-90] (br1.north);
  \draw[line width=.3pt,draw=white] (br3.east) to[out=180,in=-90] (br3.north);

  \draw[line width=.3pt,draw=white] (tr1.south) to[out=90,in=0] (tr1.west);
  \draw[line width=.3pt,draw=white] (tr3.south) to[out=90,in=0] (tr3.west);

  \draw[line width=.3pt,draw=white] (tl1.east) to[out=180,in=90] (tl1.south);
  \draw[line width=.3pt,draw=white] (tl2.east) to[out=180,in=90] (tl2.south);
  \draw[line width=.3pt,draw=white] (tl3.east) to[out=180,in=90] (tl3.south);
  

  \foreach \n in {tl1, tl2, tl3, tr1, tr3}{
    \node at (\n.east) [attach]{};
    \node at (\n.west) [attach]{};
    \node at (\n.south) [attach]{};
  }
  
  \foreach \n in {bl1, bl2, bl3, br1, br3}{
    \node at (\n.east) [attach]{};
    \node at (\n.west) [attach]{};
    \node at (\n.north) [attach]{};
  }

%---------------------------%
% Basis-Changing Graph
%---------------------------% 

\begin{scope}[xshift=14.25 cm, yshift= 1.5cm]
  \node [had] at (0.8,0.5) {$B$};
  \foreach \y in {0,1}{
  \foreach \z in {-.15,1.75}{
  \node [attach] at (\z cm, \y cm) {};}}
\end{scope}

%--------------------------%
% Hadamard Gates
%--------------------------%
\begin{scope}[xshift=-15.75 cm]
\foreach \x in {0,10,20,30}{
\begin{scope}[xshift= \x cm, yshift= -1cm]
  \node [had] at (0.8,0.5) {$H$};
  \foreach \y in {0,1}{
  \foreach \z in {-.2,1.8}{
  \node [attach] at (\z cm, \y cm) {};}}
\end{scope}}
\end{scope}

%--------------------------%
% Phase-Gate
%--------------------------%

\begin{scope}[xshift=-15.75 cm, yshift = 4 cm]
  \node [had] at (0.8,0.5) {$T$};
  \foreach \y in {0,1}{
  \foreach \z in {-.08,1.68}{
  \node [attach] at (\z cm, \y cm) {};}}
\end{scope}
\end{tikzpicture}
\end{figure}    
    
    
  \end{block}

\end{column}
\begin{column}{\sepwid}\end{column}



\begin{column}{\onecolwid}
  \begin{block}{Truncated Gaussian wavepackets}
Two key technical theorems used in \cite{MPQW} describe the bounds on time evolution of wavepackets in single particle scattering and two particle scattering.  In particular, \cite{MPQW} gave explicit bounds on the difference between the time evolution of an initial square wave packet, and the translation of the wavepacket that moves with the correct speed and has the ideal scattering behavior.  Essentially, they approximated the time evolution by the wavepacket being translated at the ideal speed.

In this manner, they were able to show that the error for a square wave packet of length $L$ grew as $\mathcal{O}(L^{-1/4})$ assuming that the total evolution time was $\mathcal{O}(L)$, and showed that the interaction of two square wavepackets moving toward each other have the ideal behavior up to the same error.

In this work, we give similar bounds for truncated Gaussian wavepackets.  In particular, let us assume that our wavepacket is a truncated Gaussian with standard deviation 
\[
  \sigma = \frac{L}{\sqrt{\log L}}
\]
truncated at a distance $L$ from the center.  If we then compare the time evolution of this wavepacket to the simple translation of the Gaussian (i.e, move the center at the correct speed for the momentum), we find that the error grows as $\OO(\sqrt{\log L/L})$.  Note that this is a near quadratic improvement over the bounds on the square wavepackets.

A similar (but slightly more complicated) proof also works for understanding two truncated Gaussians traveling toward each other.  Assuming both wavepackets have the same truncation distance and standard deviation, the time evolution of two initially separated wavepackets is well approximated by the translation of the two wavepackets at the correct speed, where the wavepacket acquires the ideal phase after the particles pass each other, up to the same error of $\OO(\sqrt{\log L/L})$.

In this manner, we show that Gaussian wavepackets have a near quadratic improvement over the proven error bounds for square wavepackets. 



    \end{block}
    
    \begin{block}{Improvements to the universal scheme}
	  As the theorems for the error bounds on the approximations to the wavepacket evolution are used in a black-box manner in the MPQW universality construction, we can essentially plug in our improvements without changing the construction at all.  In particular, the construction for universality showed that the total error bound for the construction grew as $\OO(\epsilon n g |H|)$, where $\epsilon$ is the error of a single scattering event, $n$ is the number of qubits in the system, $g$ is the number of gates in the circuit, and $|H|$ is the norm of the interaction Hamiltonian on $n$ particles.  
	
	  In particular, using the truncated Gaussian wavepackets, we have that each scattering event has essentially a quadratically better error bounds, which leads to a large improvement in the bound for the total error of the sequence of scattering events.  If we want the final construction to have constant error probability, we then have that by taking 
	  \[
	    L = \Theta \left( g^2 n^2 |H|^2 \log(gn)\right),
	  \]
	  the resulting construction will work with constant probability.  This then results in a graph of size
	  \[
	    \OO \left(g^3 n^3 |H|^2 \log(gn) \right)
	  \]
	  and a total evolution time of
	  \[
	    \OO \left( g^3 n^2 |H|^2 \log(gn)\right).
	  \]
	  These should be compared to the results of \cite{MPQW}, which yielded graphs of size $\OO(g^5n^5|H|^4)$ and evolution times of $\OO(g^5n^4|H|^4)$.
    
    
    \end{block}
  
  \begin{block}{References}
  \nocite{*}
  \bibliographystyle{myhamsplain}
  \footnotesize{\bibliography{improved_bounds_poster}}
  \end{block}


\end{column}


\begin{column}{\sepwid}\end{column}

\end{columns}
\end{frame}



\end{document}